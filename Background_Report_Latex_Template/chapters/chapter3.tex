\chapter{Novel Work}

In this section I am going to present the work I have completed so far. Two main areas of the systematic review process has been focused on. Stopping criteria and indexing/querying pubmed.

\section{Sample Method to Stopping}

As approach to determining when to stop looking at document abstracts returned by the query we are proposing a new sampling method. The first step of this method is to randomly sample a returned set of documents into a subsets.

\begin{equation}
U = \frac{|D|}{S}
\end{equation}

Where $U$ is the computed randomised subset, $D$ is the document collection and $S$ is the sample size.

We then use this subset $U$ to create a model / baseline for our topic as a way of predicting how many documents one would need to look at to hit a threshold. The intuition behind this approach is that the rate of which relevant documents occur should be relatively similar when the number of returned documents in the same.

