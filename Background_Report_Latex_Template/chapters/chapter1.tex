\chapter{Introduction}

Medical literature poses interesting challenges for Natural Language Processing (NLP) researchers. The sheer volume of medical data makes it difficult for humans to process efficiently.

Evidence-based medicine has become an important aspect in health care and policy making. One key task is the creation of systematic reviews. Systematic reviews are transparent reviews that aim to pull together and critically analyse and summarise relevant literature to a topical question. The process of creating a systematic review is rigorous and time consuming with varying degrees of complexity in-between steps. This report will look at the existing  work done using NLP as part of the systematic review process as well as the novel work done by myself so far.

We first review the stages involved in creating a systematic review. We break the steps down by looking at the PICO strategy \cite{pico}. By breaking the steps down, it becomes easier to examine potential candidates for applying NLP techniques to the process. Areas for research are then identified.

We then move on to look at stopping methods for systematic reviews. Stopping methods are about finding a suitable stopping point given a list of ranked documents. Two stopping existing methods are examined; the target method and the knee method. 

The work completed so far is then presented. We look at using curves to make predictions of finding a stopping point, including using a Gaussian process. We then present our work for the automatic query generation process by using systematic review protocols as a basis for inferring the query.


