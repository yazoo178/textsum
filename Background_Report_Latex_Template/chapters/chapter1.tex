\chapter{Introduction} \label{intro}

The problem of handling medical literature poses an interesting challenges for Natural Language Processing (NLP) researchers. The sheer volume of medical data makes it difficult for humans to process efficiently.

Evidence-based medicine has become an important aspect of health care and policy making. One key task is the creation of systematic reviews. Systematic reviews are transparent reviews that aim to pull together and critically analyse and summarise relevant literature to a topical question \cite{Gough2012}. The process of creating a systematic review is rigorous and time consuming with varying degrees of complexity in-between steps \cite{O’Mara-Eves2015}. Therefore the challenge of applying NLP to the systematic review process is being able to improve efficiency, but not compromise on rigour and reliability. 

This report will look at the existing  work done using NLP as part of the systematic review process as well as the novel work done during the early stages of this PhD.

We first review the stages involved in creating a systematic review \ref{stepssr}. We break the steps down by looking at the PICO strategy \cite{pico}, standing for patient population, intervention or exposure, comparison or control and outcome. By breaking the steps down, it becomes easier to examine potential candidates for applying NLP techniques to the process. Areas for research are then identified.

We then move on to look at stopping methods for systematic reviews. \ref{stops} Stopping methods are about finding a suitable stopping point given a list of ranked documents. Two existing stopping methods are examined; the target method and the knee method. 

In the next section we identify some relevant research questions within the field \ref{rq}. We focused the majority of the research on stopping criteria-techniques for maximizing cost effectiveness in a set of rankings. We also looked at information extraction from studies, which has potential use in further developing stopping methods.

The work completed so far is then presented \ref{novelw}. We first look at the oracle (best possible results) for our dataset. We then move on to presenting a new method that uses the similarity between the documents in the rankings to find a stopping point. We move on to using sampling methods to infer the general distribution of data and make predictions as the remainder. 

Finally we look at future work that will be undertaken for the next 2 years of the PhD. \ref{fw}.


