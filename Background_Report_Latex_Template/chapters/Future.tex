\chapter{Future Work} \label{fw}

This chapter will outline a plan for future work. This chapter will relate back to our research questions chapter. \ref{rq}

We will look at feature extraction for systematic reviews. This will involve looking at content within a study and attempting to identify it automatically. The first step of this process is being able to obtain relevant studies for a systematic review. We will look at techniques for obtaining these in the form of pdf files. Work will also need undertaking in processing a diverse range of pdf documents.

We will investigate classification of studies and determine if we can make a binary decision on whether or not a study is relevant to a research question. To achieve this we will first need to process the content of the studies and decide on a sampling strategy. This might involve using a proportion of the relevant studies as a training examples, and then trying to classify the remainder of the studies. We will also look at using unsupervised methods, by trying to directly use the systematic review as an indicator of relevant studies. We will need to overcome the challenge of condensing large studies (e.g full pdf texts) into relevant chunks.

Finally we will look at alternate approaches to finding stopping points. We found that fitting a GP caused too much over fitting to our sampled rankings. We will look at alternate regression-based machine learning algorithms to finding a stopping point.




