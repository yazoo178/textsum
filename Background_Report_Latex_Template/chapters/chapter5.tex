\chapter{Conclusions}

We first introduced the \#happysheffield web app and discussed how it is currently implemented.

We evaluated the performance of \#happysheffield using evaluation semantics. We can use this data as a base line for when we attempt to improve the performance of the web app using different affect analysis techniques.

We have revised the relevant literature that surrounds affect analysis. We have looked at different techniques we can apply, including their pros and cons. We decided which of these techniques would be suitable to apply to \#happysheffield. It was decided that we would favour supervised machine learning techniques, due to the large availability of datasets and scalability 

We looked at other potential improvements to \#happysheffield and decided to implement some extra features such as being able to compare emotion across places and displaying tweets as they arrive live from Twitter.

Finally we formally planned the project using a gnatt chart and risk assessment; identifying potential hazards and what action to take if they occur.

Overall the main objective of this project is to evaluate \#happysheffield, with a strong focus on applying modern affect analysis techniques in hope of improving the performance.



